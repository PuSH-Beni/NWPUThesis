%%==================================================
%% chapter04.tex for NWPU Bachelor Thesis
%% Encoding: UTF-8
%%==================================================

\chapter{实现}
\label{chap:Implementation}


\section{概述}
\label{sec:implementcomment}
之前的章节阐述了上层模块的逻辑设计,包含了FAT文件系统的结构及参数信息\footnote{参见第\ref{chap:fat}章},
本章将具体描述文件接口上层模块的具体实现(Implementation)方法。

对于上层模块,主要担负着完成项目要求的各项功能的任务,是整个工程最重要的也是最复杂的部分,是文件接口的核心。
同时,鉴于其与底层驱动模块以及底层硬件的联系不很紧密,本模块具有很好的可移植性和拓展性。
接下来,本节将具体描述FAT32文件接口的构建过程,出于对易读性的考虑,本节力求减少源代码的使用,
对于比较繁杂的函数尽量使用自然语言进行算法描述。
\section{前期准备}
\label{sec:prepare}
在开始描述具体的算法之前,有必要首先介绍本模块用到的数据类型,它们将在之后的函数中发挥重要的作用,
同时可以省很多不必要的麻烦,在模块化的设计过程中,使用合理的数据结构及可复用的函数模块能够减少思维的复杂度,
简化设计。

本工程定义了以下长度的数据类型以满足可移植性的要求,鉴于「int」类型在不同CPU上有不同的定义长度,
故不使用。具体定义如下:
\begin{lstlisting}[language={C}, caption={数据类型重定义}]
//长8位
typedef unsigned char	BYTE;
//长16位
typedef unsigned short	WORD;
//长32位
typedef unsigned long	DWORD;
\end{lstlisting}

接下来可以定义所需的结构体了。由于本工程不是很复杂,所以文件系统(FatFS)的结构体中除了包含FAT32文件体统参数信息之外,
还包含了文件的信息(比如文件指针、文件大小、起始簇号等)。
\begin{lstlisting}[language={C}, caption={文件系统的结构体}]
typedef struct {
	BYTE	fs_type;        //FAT文件系统的类型(FAT12、FAT16或FAT32)
	BYTE	csize;          //簇大小(单位为扇区)
	DWORD	n_fatent;       //FAT表的总记录数(卷可用簇数加2)
    DWORD   fatsize;        //单张FAT表的大小
	DWORD	fatbase;        //首张FAT表的起始地址(单位为扇区)
	DWORD	dirbase;        //根目录区起始地址(单位为扇区)
	DWORD	database;       //数据区起始地址(单位为扇区)
    //FSINFO项
    WORD    fsi_sec;        //FSINFO 的扇区号
    DWORD   freecont;       //空闲簇数
    DWORD   nxtfree;        //下个空闲簇号
    // 文件项
    BYTE    flag;
    DWORD	fptr;           //文件的读写指针
	DWORD	fsize;          //文件大小
	DWORD	org_clust;      //文件起始簇号
	DWORD	curr_clust;     //文件当前簇号
	DWORD	curr_sect;		//文件当前扇区号
} FATFS;

\end{lstlisting}
除此外,还定义了目录项(Directory Entry)的结构体。
\begin{lstlisting}[language={C}, caption={目录项的结构体}]
typedef struct {
	WORD	index;          //该目录项在目录表的序号
	BYTE	fn[12];         //文件名
	DWORD	sclust;         //目录表的起始簇号
	DWORD	clust;          //目录表的当前簇号
	DWORD	sect;           //目录表的当前扇区号
} DIR;
\end{lstlisting}
\section{FAT表的相关操作}
\label{sec:fatfunc}
在第\ref{chap:fat}章提到过的算法\ref{algo:getfat}和算法\ref{algo:putfat}分别描述了从FAT表中读取簇号和写入簇号。
这两个基础的算法在后面会很常用,凡是需要对FAT表进行操作都需要使用。

算法\ref{algo:creatchain}和算法\ref{algo:rmchain},
分别描述了如何新建和删除一个簇链。
\begin{balgo}{创建簇链}{algo:creatchain}
\begin{algorithmic}[1]
\Require $Clst_a$
\Ensure $Clst_r$
\If{$Clst_a$ is invalid}
    \State \Return Error: Invalid clusters number
\ElsIf{Create a new cluster chain}
    \If{FSI\_NxtFree is invalid}
        \State Search the EMPTY clst from \#2
    \Else
        \State Search the EMPTY clst from FSI\_NxtFree
    \EndIf
    \State Put EOC as the value of $Clst_r$ in FAT
\ElsIf{Stretch the $Clst_a$ }
    \State Get the last clst of the current chain
    \If{FSI\_NxtFree is invalid}
        \State Search the EMPTY clst from \#2
    \Else
        \State Search the EMPTY clst from FSI\_NxtFree
    \EndIf
    \State Put EOC as the value of $Clst_r$ in FAT
    \State Put $Clst_r$ as the value of $Clst_a$ in FAT
\EndIf
\State Update FSINFO struct
\State\Return $Clst_r$
\end{algorithmic}
\end{balgo}

算法\ref{algo:creatchain}需要同时考虑到创建一个新的簇链,和在给定一个簇号的情况下续接其所在的簇链。
至于删除簇链,只需沿着起始簇号删除沿途经过的簇号即可。
\begin{balgo}{删除簇链}{algo:rmchain}
\begin{algorithmic}[1]
\Require $Clst_a$
\If{$Clst_a$ is invalid}
    \State \Return Error: Invalid clusters number
\Else
    \Repeat
    \State Put the EMPTY as the value of the current clst in FAT
    \State Foward the clst pointer of the chian
    \Until{Get the last clst of the chain}
    \State Put the EMPTY as the value of the last clst in FAT
    \State Update FSINFO struct
\EndIf
\end{algorithmic}
\end{balgo}

\section{目录相关操作}
\label{sec:dirfunc}

本节将对涉及目录操作的函数做一说明。算法\ref{algo:dirfind}描述了在一个目录表\footnote{通常是根目录}中检错目录项的过程。
检索时要注意目录表是否是跨簇的长表,这要求在将指针向后移动的时候要做出检查。
鉴于一个簇总是包含2的整数次幂个扇区,故首先判断指针是否在扇区边界,然后在判断是否在簇的边界。
如果在扇区边界,要与硬盘交换扇区(不能跨扇区地址读写数据);如果在簇边界,要通过FAT表循簇链找到下一块簇,
再将该簇的首扇区读入内存。
\begin{balgo}{检索目录项}{algo:dirfind}
\begin{algorithmic}[1]
\Require DIR $*dp$
\If{$dp$ is invalid}
    \State \Return Error: Invalid directory
\Else
    \State Rewind the $*dp$ \Comment{Move the dir-pointer to the first entry of the directory table}
    \Repeat
    \State Get the DIR\_Name of the current dir-entry
    \State Foward the dir-pointer of the table
    \Until{This DIR\_Name matches dp->fn[]}
    \If{The dir-pointer out of range}\Comment{Mismatch}
        \State \Return Error: Not found
    \Else
        \State \Return Match
    \EndIf
\EndIf
\end{algorithmic}
\end{balgo}

算法\ref{algo:dirreg}描述了在目录表(Directory Table)中注册一个新的目录项(Directory Entry)的过程。
新建时要先找到一块空闲的记录项(Empty Entry),检测其DIR\_Name[0]是否为「DDEM」或「0」即可,
具体可见第\ref{sec:dirstruct}节。
找到空闲记录项后,在FAT表新建一个簇链,并将取得的起始簇号、目录名等信息写入该目录项即可。
\begin{balgo}{注册目录项}{algo:dirreg}
\begin{algorithmic}[1]
\Require DIR $*dp$
\If{$dp$ is invalid}
    \State \Return Error: Invalid directory
\Else
    \State Rewind the $*dp$
    \Repeat
    \State Get the DIR\_Name[0] of the current dir-entry
    \State Foward the dir-pointer of the table
    \Until{This DIR\_Name[0] matches DDEM or 0x00}
    \If{The dir-pointer out of range}\Comment{Mismatch}
        \State \Return No free entry
    \Else
        \State Creat a new cluster chain in FAT, get the $Clst_r$
        \State Store the $Clust_r$ and dp->fn[] in entry
    \EndIf
\EndIf
\end{algorithmic}
\end{balgo}

\section{文件相关操作}
\label{sec:filfunc}

首先进行文件系统挂载的函数说明,算法\ref{algo:mountfs}描述了该过程。
该算法重点在于DBR扇区的正确加载。物理上的「0」号扇区一般是MBR记录,其在偏移0x1C6开始的4个字节中,记录了DBR的起始扇区。
需要再次强调的是,磁盘上的存储字节的方式都是小端模式,有时可能需要转换。
通过之前的算法\ref{algo:cntofclus}、算法\ref{algo:voltype}、算法\ref{algo:fatent},
可以得到文件系统的相关信息,并记录在$fs$所指向的文件系统数据类型中。
\begin{balgo}{挂载文件系统}{algo:mountfs}
\begin{algorithmic}[1]
\Require FATFS $*fs$
\State Check if physical drive is ready or not
\State Load $0^{th}$ sector if okay
\If{Offset 510 and offset 511 $\neq$ 0xaa55}
    \State \Return Error: Not a boot sector
\ElsIf{The first byte of $0^{th}$ sector $\neq$ 0xEB}
    \State Get the dbr sector number which read at offset 0x1C6
    \State Load the dbr sector
\Else 
    \State This sector is dbr sector
\EndIf
\State Load the information from bpb,bs,fsinfo struct

\end{algorithmic}
\end{balgo}

在挂载文件系统之后,$*fs$会作为全部变量存在,即该文件系统的指针始终可用,知道程序终结,
故之后的函数都可以调用这个指向文件系统的指针。

接下来就是打开文件(创建文件)操作的函数,见算法\ref{algo:fopen}。
如果在根目录表中找不到给出的文件名,说明该文件不存在,此时需要新建一个文件。
通过算法\ref{algo:dirreg}在根目录表中新建一个包含该文件名的目录项,之后更新各个数据结构中的信息。
需注意,每次新打开文件后,文件指针都为零。
\begin{balgo}{打开文件}{algo:fopen}
\begin{algorithmic}[1]
\Require CONST BYTE $*name$,\quad DIR $*dp$
\If{File system not mounted}
    \State \Return Error: Not mounted
\EndIf
\If{$*name$ Including invalid characters}
    \State \Return Error: Invalid file name
\Else
    \State Change it to proper type and store it in buffers \Comment{Must be UPPERCASE}
\EndIf
\If{Not found the $*name$ in root directory}
    \State Register an entry in directory table
    \If{Not found the $*name$}
        \State \Return Error: Register error
    \EndIf
\EndIf
\State Update the relative information of $*fs$
\end{algorithmic}
\end{balgo}

文件打开后就可以通过文件读写指针来读写文件了,分别如算法\ref{algo:fread}和\ref{algo:fwrite}所示。
读文件和写文件操作正好相反,整体类似,都需要注意跨扇区和跨簇的问题,前者只需要重新加载一个新扇区即可,
但是跨簇时要通过簇链找到下一块簇,并加载该簇的第一个扇区。
\begin{balgo}{读文件}{algo:fread}
\begin{algorithmic}[1]
\Require BYTE $*buff$,\quad DWORD $btr$ \Comment{Read-buffers and the number of bytes to read}
\If{File not opened}
    \State \Return Error: Not opened
\EndIf
\Repeat
\If{The file r/w pointer on the sector boundary}
    \If{The file r/w pointer on the cluster boundary}
        \State Update the current cluster number by cluster chain from FAT
    \Else
        \State Update the current sector number
    \EndIf
\EndIf
\State Load the current sector to buffers
\State Read data to $*buff$
\State Update the r/w pointers and $btr$
\Until{All bytes have been transfered}
\end{algorithmic}
\end{balgo}

对于写操作,还要注意更新文件大小信息,并且,当簇链循到了最后一个簇号时还要新建簇号以续接簇链。
\begin{balgo}{写文件}{algo:fwrite}
\begin{algorithmic}[1]
\Require BYTE $*buff$,\quad DWORD $btw$ \Comment{Write-buffers and the number of bytes to write}
\If{File not opened}
    \State \Return Error: Not opened
\EndIf
\If{The file size will be greater after write operation}
    \State Update the file size
\EndIf
\Repeat
\If{The file r/w pointer on the sector boundary}
    \If{The file r/w pointer on the cluster boundary}
        \State Update the current cluster number by cluster chain from FAT
        \If{Meet the last cluster of the chain}
            \State Join a new empty cluster to the chain
        \EndIf
    \Else
        \State Update the current sector number
    \EndIf
\EndIf
\State Load the current sector to buffers
\State Write the data from $*buff$ to sectors
\State Update the r/w pointers and $btr$
\Until{All bytes have been transfered}
\end{algorithmic}
\end{balgo}

最后,文件的操作还涉及到了删除文件。
如算法\ref{algo:frm}所示,删除文件相对而言比较简单,只需清空目录项(Directory Entry),
然后在FAT表中删除簇链,最后更新FSINFO数据结构即可。
\begin{balgo}{删除文件}{algo:frm}
\begin{algorithmic}[1]
\Require CONST BYTE $*name$
\If{File system not mounted}
    \State \Return Error: No file system
\EndIf
\If{$*name$ Including invalid characters}
    \State \Return Error: Invalid file name
\Else
    \State Change it to proper type and store it in buffers
\EndIf
\If{Not found $*name$ in the directory}
    \State \Return Error: No such file
\EndIf
\State Clear the directory entry in table (set DDEM in DIR\_Name[0])
\State Remove the cluster chain in FAT
\State Update the FSINFO
\end{algorithmic}
\end{balgo}


\section{本章小结}
\label{sec:sum4}
本章详细阐述了上层模块的具体实现过程,
重点描述了相关的函数接口的具体算法实现,出于简洁性和易读性的考虑,
全部采用自然语言的描述方法,同时,在第\ref{chap:fat}章出现过的算法本章讲不在赘述。
