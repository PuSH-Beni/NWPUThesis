\begin{grasum}
    历时半年的本科毕业设计终于告一段落。

    在仔细地学习研究并理解FAT官方文档后,我终于能够亲手做出FAT32的文件接口了,尽管有些地方还不够完善,
    但毕竟实现了预期的功能,我还是比较满意的。在这半年中,我通过网路查阅了大量资料,
    当然,主要的、根本的参考资料是官方文档(分别来自Microsoft Coporation 和 SD Group),
    上百页的资料都需要仔细阅读、理解。
    同时我还参考了@ChaN的FatFs开源项目,学习到了文件系统的整体设计思路,获益匪浅。

    通过毕业设计的锻炼,在具体实践的过程中,我对FAT文件系统和SD卡都有了深刻的了解,
    尤其是对文件系统有了更本质、更底层、更具体的理解。并且,通过完成实际项目,我也认识到了理论与实践的距离,
    完美的理论也需要可操作的实验验证,上层设计终究要与底层硬件模块相结合才能实现所需要的功能。
    
\end{grasum}
