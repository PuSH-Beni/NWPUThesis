%%==================================================
%% chapter02.tex for NWPU Bachelor Thesis
%% Encoding: UTF-8
%%==================================================

\chapter{设计}
\label{chap:Design}
本章将具体描述FAT32文件系统接口的整体架构和设计思路。
从要求实现的功能出发,将本接口分为两个模块,并为每个模块都设计符合规定的应用程序接口(API)以供上层程序调用。

本章将阐述上层模块,即FAT文件系统部分,作为整个工程中的一个子模块的逻辑设计过程。
在详细说明FAT文件系统组成结构的基础上,简要描述部分算法以便理解。
本模块为顶层用户程序提供的API如下:
\begin{lstlisting}[language={C}, caption={上层模块的API}]
// FAT module application interface
FRESULT f_mount (FATFS* fs, BYTE *sd_type);                         //FAT文件系统的挂载
FRESULT f_open (const char* name, DIR* dj);                         //打开文件
FRESULT f_close(void);                                              //关闭文件
FRESULT f_read (BYTE* buff, DWORD btr, DWORD* br);                  //读文件到缓存
FRESULT f_write (const BYTE* buff, DWORD btw, DWORD* bw, DIR* dj);  //从缓存写文件
FRESULT f_seek_ptr (DWORD ptr);                                     //重设文件中的读写指针
FRESULT f_rm (const char* name, DIR* dj);                           //删除文件
\end{lstlisting}
整个FAT文件系统最初是基于IBM的PC架构而研发的,由此有一个很重要的特性就是,
FAT文件系统在磁盘上的数据是以小端模式\footnote{与此相对的,还存在另一种存储方式,
即大端模式(Big Endian)}(Little Endian)存储的。
\begin{table}[!htb]
  \centering
  \caption[小端模式存储示例]{32位数据在小端模式下的结构}\label{tab:littleendian}
  \begin{tabular}{lcr} \toprule
      byte[3] & &3 3 2 2 2 2 2 2 \\
              & &1 0 9 8 7 6 5 4 \\
      byte[2] & &2 2 2 2 1 1 1 1 \\
              & &3 2 1 0 9 8 7 6 \\
      byte[1] & &1 1 1 1 1 1 0 0 \\
              & &5 4 3 2 1 0 9 8 \\
      byte[0] & &0 0 0 0 0 0 0 0 \\
              & &7 6 5 4 3 2 1 0 \\ \bottomrule
  \end{tabular}
\end{table}
一个FAT文件系统卷由四个基本区域组成,它们顺序排列如下:
\begin{enumerate}[noitemsep,topsep=0pt,parsep=0pt,partopsep=0pt,leftmargin=6cm]
    \item[0) ––] 保留区(Reserved Region)
    \item[1) ––] FAT表区(FAT Region)
    \item[2) ––] 根目录区\footnote{FAT32文件系统卷不存在根目录区}(Root Directory Region)
    \item[3) ––] 数据区(File and Directory Data Region)
\end{enumerate}
\section{启动扇区和BPB}     %offsets
\label{sec:bsbpb}
FAT卷上第一重要的数据结构被称作BPB(BIOS Parameter Block),它位于保留区域(Reserved Region)的第一个扇区。
该扇区有时又被称为启动扇区(Boot Sector)、保留扇区(Reserved Sector)或零号扇区($0^{th}$ Sector),
事实上,它是整个FAT卷的第一个扇区。

启动扇区的具体结构如表\ref{tab:bsbpb}所示,其中需要注意的是,名字以「BPB\_」开头的项是BPB的一部分,
而以「BS\_」开头的项是Boot Sector的一部分,并不属于BPB。
\begin{longtable}[!htb]{lccp{7cm}}
\caption{启动扇区与BPB结构} \label{tab:bsbpb}\\
    \toprule
    \makecell[c]{名称} & 偏移(byte) & 长度(bytes) & \makecell[c]{描述} \\ \midrule
    BS\_JumpBoot & 0 & 3 & 跳向启动代码,常用值是「0xEB」。\\ \midrule
    BS\_OEMName & 3 & 8 & 这只是一个名称字符串,推荐值是「MSWIN4.1」。\\ \midrule
    BPB\_BytsPerSec & 11 & 2 & 扇区字节数,仅可从「512」、「1024」、「2048」或「4096」中取值。
    推荐值为「512」,此时具有最好的兼容性。\\ \midrule
    BPB\_SecPerClus & 13 & 1 & 每簇扇区数,取值必须为2的次幂,并且大于0。合法值为「1」、「2」、
    「4」、「8」、「16」、「32」、「64」及「128」。然而要注意的是,
    每簇字节数(BPB\_BytsPerSec * BPB\_SecPerClus)不能超过32K(32 * 1024)。\\ \midrule
    BPB\_RsvdSecCnt & 14 & 2 & 保留区域的扇区数目,必须为非零值。FAT12和FAT16必须为「1」,
    FAT32典型值为「32」。\\ \midrule
    BPB\_NumFATs & 16 & 1 & FAT表在磁盘卷上的数目,通常设为「2」。\\ \midrule
    BPB\_RootEntCnt & 17 & 2 & 对FAT12和FAT16来说,这是根目录能够容纳的目录/文件记录(32B-Entries)的数目。
    对FAT32而言,此值域必须为「0」。\\ \midrule
    BPB\_TotSec16 & 19 & 2 & 卷中的总扇区数,对FAT32卷此域必须为「0」。若此域为「0」,
    BPB\_TotSec32必须非零。 \\ \midrule
    BPB\_Media & 21 & 1 & 「0xF8」是非可插拔媒介的标准值,「0xf0」是可插拔媒体的常用值。\\ \midrule
    BPB\_FATSz16 & 22 & 2 & 对FAT12和FAT16有效,是\textbf{单张}FAT表大小(所占扇区数)。FAT32此域必须为「0」,
    同时BPB\_FATSz32必须非零。\\ \midrule
    BPB\_SecPerTrk & 24 & 2 & 磁道扇区数 \\ \midrule
    BPB\_NumHeads & 26 & 2 & 磁头数 \\ \midrule
    BPB\_HiddSec & 28 & 4 & 隐藏扇区数,位于包含文件系统卷的分区之前。无分区则设为「0」。\\ \midrule
    BPB\_TotSec32 & 32 & 4 & FAT32文件系统卷的总扇区数。\\
    \bottomrule
\end{longtable}

接下来的表\ref{tab:fat32off36}是FAT32文件系统从Offset 36开始的数据结构。注意,FAT12/FAT16和FAT32在这之后的参数是不同的,
它们也有自己的一张表格,限于篇幅就不列出了。
\begin{longtable}[!htbp]{lccp{7cm}}
\caption{FAT32在Offset 36之后的参数} \label{tab:fat32off36} \\
    \toprule
    \makecell[c]{名称} & 偏移(byte) & 长度(bytes) & \makecell[c]{描述} \\ \midrule
    BPB\_FATSz32 & 36 & 4 & \textbf{单张}FAT表大小(扇区数),仅对FAT32有效。\\ \midrule
    BPB\_ExtFlags & 40 & 2 & Bits 0-3 ––– 激活的FAT表数目,从零开始。\\
                  &   &    & Bits 4-6 ––– 保留。\\
                  &   &    & Bits\hspace{1em}7 ––– 「0」表示FAT表随时进行备份;「1」表示只有一个FAT表被激活。\\
                  &   &    & Bits 8-15 –– 保留。 \\ \midrule
    BPB\_FSVer & 42 & 2 & FAT32卷的版本号。\\ \midrule
    BPB\_RootClus & 44 & 4 & FAT32的根目录的起始簇号,通常设置为「2」。\\ \midrule
    BPB\_FSInfo & 48 & 2 & FSINFO结构在保留区的扇区号,通常设为「1」。\\ \midrule
    BPB\_BkBootSec & 50 & 2 & 如果此域非零,则表示启动记录在保留区的备份扇区号,推荐值为「6」。\\ \midrule
    BPB\_Reserved & 52 & 12 & 未来扩展的保留域,格式化时必须全部置零。\\ \midrule
    BS\_DrvNum & 64 & 1 & 设备编号,硬盘可设为「0x80」,软盘「0x00」。\\ \midrule
    BS\_Reserved1 & 65 & 1 & 用于Windows NT。格式化时置零。\\ \midrule
    BS\_BootSig & 66 & 1 & 扩展的启动标识。\\ \midrule
    BS\_VolID & 67 & 4 &  FAT文件系统卷编号,常和BS\_VolLab一起用来跟踪可移动媒介。 \\ \midrule
    BS\_VolLab & 71 & 11 & 卷标,无卷标时此域被设置为「\textbf{NO NAME}\qquad\qquad」。\\ \midrule
    BS\_FilSysType & 82 & 8 & 设置为字符串:「\textbf{FAT12}\qquad\quad」,「\textbf{FAT16}\qquad\quad」,
    「\textbf{FAT}\qquad\qquad\quad」中的一个。\\
    \bottomrule
\end{longtable}

另外一个很重要的点是,对于FAT卷的第一块扇区,即Boot Sector($0^{th}$ Sector),如果将其内容以字节为单位当成数组的话,
那么Sector[510]等于「0x55」,Sector[511]等于「0xAA」。

\section{FAT数据结构}
\label{sec:datastruct}

在BPB之后的另一个重要的数据结构就是FAT数据结构本身。FAT数据结构为每个文件都定义了一个单链表,链接着文件在数据区所占据的每一个簇号。
这样看来,目录(Directory)也不过是一种文件,只不过它有一些参数用来被文件系统识别成目录。
FAT表是通过簇号(Cluster Number)来映射文件卷(Volume)上的数据区(Data Region)的,第一个可用簇号是「2」号簇。

「2」号簇的第一个扇区将通过文件卷的BPB域的一些参数计算得出,同时,这个扇区也是数据区的第一个扇区。
具体如算法\ref{algo:firstsec}所示。
\begin{balgo}{求数据区的第一个扇区}{algo:firstsec}
\begin{algorithmic}
\Ensure $FATSz$, $FirstDataSec$
\If{$BPB\_FATSz16 \neq 0$}
\Comment{判断单张FAT表的大小由哪个域决定}
    \State $FATSz \gets BPB\_FATSz16$
\Else
    \State $FATSz \gets BPB\_FATSz32$
\EndIf
\State $FirstDataSector \gets BPB\_ResvdSecCnt + (BPB\_NumFATs \times FATSz) + RootDirSectors$
\Comment{FAT32的RootDirSectors等于零}
\end{algorithmic}
\end{balgo}

需要注意的是,这里的扇区号是相对于包含BPB的文件卷(Volume)的第一个扇区(Sector 0)而言的,
毕竟文件卷的第一个扇区不一定是设备(Drive)的第一个扇区(磁盘分区会产生影响)。


\section{FAT类型定义}  %FAT12,16,32,FAT_TABLE
\label{sec:fattype}

FAT文件系统的类型(「FAT12」、「FAT16」、「FAT32」)仅仅取决于文件卷上的簇数,而与其他参数无关。
通过算法\ref{algo:cntofclus},可以计算出卷中的簇数。
\begin{balgo}{求文件卷的簇数}{algo:cntofclus}
\begin{algorithmic}
\Require $FATSz$
\Ensure $CountofClusters$
\If{$BPB\_TotSec16 \neq 0$}
    \State $TotSec \gets BPB\_TotSec16$
\Else
    \State $TocSec \gets BPB\_TotSec32$
\EndIf
\State $DataSec \gets TotSec - (BPB\_ResvdSecCnt + BPB\_NumFATs \times FATSz + RootDirSectors)$
\Comment{FAT32的RootDirSectors等于零}
\State $CountofClusters \gets DataSec \div BPB\_SecPerClus$
\end{algorithmic}
\end{balgo}

接下来可以由$CountofClusters$得到FAT卷的类型(算法\ref{algo:voltype})。
\begin{balgo}{求文件卷的类型}{algo:voltype}
\begin{algorithmic}
\Require $CountofClusters$
\Ensure $FATType$
\If{$CountofClusters < 4085$}
    \State $FATType \gets FAT12$  \Comment{$Volume\ is\ FAT12$}
\ElsIf{$CountofClusters < 65525$}
    \State $FATType \gets FAT16$ \Comment{$Volume\ is\ FAT16$}
\Else
    \State $FATType \gets FAT32$ \Comment{$Volume\ is\ FAT32$}
\EndIf
\end{algorithmic}
\end{balgo}

这里的$CountofClusters$是指数据区(Data Region)的总簇数,不过由于数据区的簇号是从「2」号簇开始计数的,
因此整个文件卷的最大可用簇号是$CountofClusters + 1$。同时,如果包含两个保留簇号(「0」号和「1」号簇),
那么所有簇号的总数应该是$CountofClusters + 2$。

接下来我们将面临一个很重要的问题,这涉及到FAT系统的基础:给定任意合法簇号$N$,FAT表中的相应簇号的记录在哪?
这对于FAT12而言比较复杂,鉴于它过于古老,我们就不涉及了。现在重点研究FAT32的计算方法。
\begin{balgo}{求FAT表的簇号记录}{algo:fatent}
\begin{algorithmic}
\Require $N$, $FATType$
\Ensure $FATSecNum$,  $FATEntOffSet$
\If{$FATType = FAT16$}
    \State $FATOffSet \gets N \times 2$
\Else
    \State $FATOffSet \gets N \times 4$
\EndIf
\State $FATSecNum \gets BPB\_ResvdSecCnt + FATOffset \div BPB\_BytsPerSec$
\State $FATEntoffSet \gets FATOffset \pmod{BPB\_BytsPerSec}$
\end{algorithmic}
\end{balgo}

在算法\ref{algo:fatent}中的$FATSecNum$是簇号记录在第一张FAT表上的扇区号,如果想得到第二张FAT表上对应记录的扇区号,
只需要对$FATSecNum$加$FATSz$即可,之后\footnote{如果该卷的FAT不止两张的话}的FAT表依此类推。

现在让我们假设,$FATSecNum$这个扇区\footnote{要注意,这个扇区号是相对于卷的第一个扇区而言的}被读进了一个名为$Buff$的字节数组。
同时再假定$WORD$是一个16位无符号数(16-bit Unsigned),$DWORD$是一个32位无符号数(32-bit Unsigned)。
我们通过算法\ref{algo:getfat}从FAT表得到该簇号记录的内容(其实是该簇号$N$所在簇链的下一个簇号)。
\begin{balgo}{求FAT表簇号记录的内容($ClusEntVal$)}{algo:getfat}
\begin{algorithmic}
\Require $FATEntOffSet$
\If{$FATType = FAT16$}
    \State $FAT16ClusEntVal \gets *((WORD*)\ \&Buff[FATEntOffSet])$
\Else
    \State $FAT32ClusEntVal \gets$
    \State $(*((DWORD*)\ \&Buff[FATEntOffset]))\quad\&$\quad0x0FFFFFFF
\EndIf
\end{algorithmic}
\end{balgo}

通过算法\ref{algo:putfat}将内容存入FAT表的相应记录。
\begin{balgo}{将内容($ClusEntVal$)存入FAT表相应记录}{algo:putfat}
\begin{algorithmic}
\Require $FATEntOffSet$
\If{$FATType = FAT16$}
    \State $*((WORD *)\ \&Buff[FATEntOffset]) \gets FAT16ClusEntryVal$
\Else
    \State $FAT32ClusEntryVal \gets FAT32ClusEntryVal\quad\&$\quad0x0FFFFFFF
    \State $*((DWORD *)\ \&Buff[FATEntOffset]) \gets$
    \State $(*((DWORD *) \&Buff[FATEntOffset]))\quad\&$\quad0xF0000000
    \State $*((DWORD *)\ \&Buff[FATEntOffset]) \gets$
    \State $(*((DWORD *) \&Buff[FATEntOffset]))\quad|\quad FAT32ClusEntryVal$
\EndIf
\end{algorithmic}
\end{balgo}

注意到$FAT32ClusEntVal$在两个算法都有「\& 0x0FFFFFFF」的操作,这是因为FAT32的FAT簇号实际只有28位可用,
高4位是保留位。高4位只有在格式化文件系统的时候才会被改变,此时,它们连同低28位一起被置零。

文件在FAT系统上是这样组织的:
在目录记录表(Directory Entry)中,文件(File)的起始簇号会被登记下来,同时,FAT表中会为它创建一条簇号链(Cluster Chain)。
之后每次要读写该文件时,只需顺着簇号链沿途索引即可,得到的每一个簇号($N$)都对应了文件在数据区(Data Region)的每一个数据簇。
通过前面的算法\ref{algo:firstsec}得到的$FirstDataSec$再加上簇号$N$在数据区的偏移即可得到该簇的第一块扇区号。
详见算法\ref{algo:clus2sec}。
\begin{balgo}{求簇号$N$的起始扇区号}{algo:clus2sec}
\begin{algorithmic}
\Require $FirstDataSec$,$N$
\State $FirstSecofClus \gets FirstDataSec + (N - 2)\times BPB\_SecPerClus$
\State\Comment{这里簇号$N$要减去「2」是因为簇号是从「2」号簇开始的,尽管在物理上「2」号簇是数据区的第一个簇。}
\end{algorithmic}
\end{balgo}

注意对空文件而言,其起始簇号在FAT表中的记录内容是$EOC$\footnote{End Of Clusterchain}标记。至于$EOC$标记(EOC Mark)的内容,
取决于FAT文件系统的类型。具体的判断$EOC$标记的方法可以参见算法\ref{algo:eoc}。
\begin{balgo}{EOC标记}{algo:eoc}
\begin{algorithmic}
\Require $FATType$
\State $IsEOF \gets FALSE$
\If{$FATType = FAT12$}
    \If{$FATContent\quad\geq\quad$0x0FF8}
    \State$IsEOF \gets TRUE$
    \EndIf
\ElsIf{$FATType = FAT16$}
    \If{$FATContent\quad\geq\quad$0xFFF8}
    \State $IsEOF \gets TRUE$
    \EndIf
\ElsIf{$FATType = FAT32$}
    \If{$FATContent\quad\geq\quad$0x0FFFFFF8}
    \State $IsEOF \gets TRUE$
    \EndIf
\EndIf
\end{algorithmic}
\end{balgo}

当FAT表记录到文件的最后一个簇号是,该簇号记录的内容也被填上$EOC$标记。微软(Microsoft)操作系统\footnote{对于不同的系统,$EOC$
标记可以有不同的值,但必须符合规定}的$EOC$标记对FAT12、
FAT16和FAT32分别是 0x0FFF、0xFFFF和0x0FFFFFFF。这里还有一个坏簇标记(BAD CLUSTER),它标记了磁盘的损坏区域,
内容分别为 0x0FF7、0xFFF7和0x0FFFFFF7。

\section{目录结构}    %dir_entries,file_entries
\label{sec:dirstruct}

FAT文件系统中的目录,实际上依然是一种文件。这种特殊的文件是由一系列32字节的目录记录(Directory Entry)所组成。
对FAT12和FAT16而言,整个文件卷存在着一个根目录(Root Directory),它紧跟在最后一张FAT表的后面。然而,
FAT32并不存在这种静态的根目录,而是一张动态的根目录链表。在BPB中仅记录了FAT32的根目录的起始簇号(通常为「2」号簇),
并且文件系统格式化时,FAT32的根目录已被分配了一个簇的空间,之后会随着根目录记录的增长而自动扩展簇链,
就如同普通目录一样。

FAT32的根目录的与普通目录的不同之处在于,根目录没有时间戳(Date or Time Stamps),没有文件名\footnote{只有隐含文件名「\textbackslash」},
并且没有「当前目录」\footnote{「\quad.\quad」文件}文件和「父目录」\footnote{「\quad..\quad」文件}文件。另外,根目录还有比较特殊的一点是,
它是文件卷上唯一一个可以合法拥有「ATTR\_VOLUME\_ID」标志位的文件,详细信息可见表\ref{tab:dirstruct}。
\begin{longtable}[!htb]{lccp{7cm}}
    \caption{FAT文件系统目录记录结构} \label{tab:dirstruct}\\
    \toprule
    \makecell[c]{名称} & 偏移(byte)& 长度(bytes)& \makecell[c]{描述} \\ \midrule
    DIR\_Name & 0 & 11 & 短文件名(SFN)\\ \midrule
    DIR\_Attr & 11 & 1 & 文件标志位:\\
              &    &   & ATTR\_READ\_ONLY ––––– 0x01\\
              &    &   & ATTR\_HIDDEN\hspace{2em} ––––– 0x02\\
              &    &   & ATTR\_SYSTEM\hspace{2em} ––––– 0x04\\
              &    &   & ATTR\_VOLUME\_ID ––––– 0x08\\
              &    &   & ATTR\_DIRECTORY\hspace{1pt} ––––– 0x10\\
              &    &   & ATTR\_ARCHIVE\hspace{1.5em} ––––– 0x20\\ \midrule
    DIR\_NTRes & 12 & 1 & Windows NT的保留字节\\ \midrule
    DIR\_CrtTimeTenth & 13 & 1 & 文件创建时间的毫秒戳 \\ \midrule
    DIR\_FstClusHI & 20 & 2 & 本记录的起始数据簇号的高16位(FAT12和FAT16此位置零)\\ \midrule
    DIR\_WrtTime & 22 & 2 & 文件的最后写入时刻\\ \midrule
    DIR\_WrtDate & 24 & 2 & 文件的最后写入日期\\ \midrule
    DIR\_FstClusLO & 26 & 2 & 本记录的起始数据簇号的低16位\\ \midrule
    DIR\_FileSize & 28 & 4 & 文件大小(单位字节)\\
    \bottomrule
\end{longtable}

关于DIR\_Name[0]有几点要注意:
\begin{itemize}[noitemsep,topsep=0pt,parsep=0pt,partopsep=0pt]
    \item 如果DIR\_Name[0]=0xE5,该记录块(32Bytes Entry)为空闲记录;
    \item 如果DIR\_Name[0]=0x00,该记录为空的同时,在它之后的所有记录块都为空;
    \item 如果DIR\_Name[0]=0x05,事实上等同与「0xE5」。
\end{itemize}

DIR\_Name[0]必须不等于「0x20」。文件名和扩展名之间的「.」是隐含的,不会出现在DIR\_Name中。
DIR\_Name中不允许出现小写字母。下面的值对DIR\_Name而言是\textbf{非法}值:
\begin{itemize}[noitemsep,topsep=0pt,parsep=0pt,partopsep=0pt]
    \item 除了在DIR\_Name[0]可以出现的「0x05」外,任何小于「0x20」的值均不合法;
    \item 「0x22」,「0x2A」,「0x2B」,「0x2C」,「0x2E」,「0x2F」,「0x3A」,「0x3B」,「0x3C」,
        「0x3D」,「0x3E」,「0x3F」,「0x5B」,「0x5C」,「0x5D」和「0x7C」均不合法。
\end{itemize}

\section{FSINFO}
\label{sec:fsinfo}

在FAT32文件系统中,FAT表可能是占据非常大空间的数据结构,在这一点上并不同于FAT16的FAT表最大128KB,
或者FAT12的FAT表最大只能是6KB。因此FAT32的每个文件卷都采取了一种措施以便在API调用时能够迅速找到空闲簇号,
即通过记录最后一个已分配的簇号的办法。「FSInfo」扇区号记录在「BPB\_FSInfo」域,对微软的操作系统而言,
该值总被设置为「1」。表\ref{tab:fsinfo}显示了「FSInfo」扇区的详细结构。
\begin{longtable}[!htb]{cccp{7cm}}
    \caption{FAT32的FSInfo扇区结构}\label{tab:fsinfo}\\
    \toprule
    名称 & 偏移 & 大小 & \makecell{c}{描述} \\ \midrule
    FSI\_LeadSig & 0 & 4 & 值必须为「0x41615252」。该引导值用来标志本扇区是FSinfo扇区。\\ \midrule
    FSI\_Reserved1 & 4 & 480 & 保留域。FAT32格式总是初始化为「0」,并且目前此域的字节都不可用。\\ \midrule
    FSI\_StrucSig & 484 & 4 & 值必须为「0x61417272」。另一个标志位。\\ \midrule
    FSI\_Free\_Count & 488 & 4 & 储存了该卷当前已知的空闲簇的总数。如果此域为「0xFFFFFFFF」,
        则空闲簇总数未知并且需要计算。该域值需作范围检测以保证其不大于文件卷的总簇数。\\ \midrule
    FSI\_Nxt\_Free & 492 & 4 & 该域指示了文件系统应该从哪个簇号开始搜索。通常该值会被设为最后一个已分配的簇号,
        如果此域的值为「0xFFFFFFFF」,那么文件系统需要从「2」号簇开始搜索。\\ \midrule
    FSI\_Reserved2 & 496 & 12 & 保留域。初始化时置零。当前不可用。\\ \midrule
    FSI\_TrailSig & 508 & 4 & 值为「0xAA550000」。指示本扇区为FSInfo扇区。\\
    \bottomrule
\end{longtable}

\section{本章小结}
\label{sec:sum2}
本章详细说明了FAT文件系统的基本信息及相关数据结构组成,对FAT的重要结构(如BPB结构、
FAT表结构等)做了比较详细的说明,对文件系统的目录结构、类型定义也进行了详细阐述,
同时对部分算法(如计算总簇数,计算FAT表起始扇区等)做了伪代码的描述,
这为之后的文件系统的实现做好了理论准备。
