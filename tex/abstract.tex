%%==================================================
%% abstract.tex for SJTU Master Thesis
%%==================================================

\begin{abstract}
安全数位卡,即SD卡(Secure Digital Memory Card),是一种存储卡,被广泛应用于便携设备上。

单片机,即单片微型计算机(Single-chip Microcomputer),是把中央处理器、存储器、定时/计数器、
各种输入输出接口等都集成在一块集成电路芯片上的微型计算机。
与应用在个人电脑中的通用型微处理器相比,它更强调自供应(不用外接硬件)和节约成本。
它的最大优点是体积小,可放在仪表内部,但存储量小,输入输出接口简单,故需要外接存储设备,SD卡就是经济高效的最佳选择。

然而51单片机并没有SD卡接口,同时,单片机从SD卡读写的数据要与计算机实现兼容性,要求其必须支持相应的文件系统。
鉴于目前计算机大多使用微软(Microsoft)的FAT32文件系统,本课题所要设计的文件接口将在支持FAT32文件系统的同时,
解决51单片机与计算机通过SD卡的通信问题。

本课题通过模块化的设计方法,自顶向下分层设计相应的子模块以实现相应部分功能。并且,
在各个模块之间仅留出下层对上层的函数接口,这样既利于子模块单独测试,又便于各个模块的移植,
同时整个工程框架清晰,层次分明,便于调试。
具体函数的设计实现,各个参数的选取设置均以官方说明文档为准。

为了令本文件接口具有良好的可移植性以及标准性,本课题在设计研究过程中参考了一些开源项目以及官方文档,
严格做到契合相关协议和规范,同时不失自己的特色,为本地差异化配置留出了足够空间。
最终,本接口实现了对FAT32文件系统的挂载识别,文件的打开关闭,创建删除以及读写功能。
对底层的SD卡具有识别能力,能够识别SDSC、SDHC/XC,能够识别「MBR」和「DBR」扇区。

通过本课题的实践,我不仅对FAT32文件系统有了深刻的认识和了解,并且对SD卡的底层协议规范有了彻底的认知,最终,
基于微处理器的文件接口的实现更是说明了在51单片机这类资源有限的微型计算机上搭建对大型文件系统(如FAT32)的接口支持
是可行的。

\keywords{ 单片机,FAT32,SD卡,文件接口 }
\end{abstract}

\begin{englishabstract}

Secure Digital cards, 
a memory card, is widely used in portable devices.

A variety of input and output interfaces are all integrated on a single integrated circuit chip microcomputers,
which is called Single-chip Microcomputer.
Its biggest advantage is small size, can be placed inside of the instrument.
While its poor amounts of storage, it requires an external storage device that SD card is a cost-effective choice.

However, the C8051 MCU have no SD card interface.
Given the mostly computer using Microsoft OS, whose file system is FAT32.
The file interfaces to be designed for this project will support the FAT32 file system at the same time,
C8051 MCU can communicate and share the data with computer by SD card.

The specific functions and parameters will be ruled by official documents.

In order to make the file interface portable and standard, 
reference is made to some open source projects as well as official documents in my design and implementation process.
Strictly to fit the relevant agreements and norms, without losing its own characteristics,
the interface set aside enough space for local specific configuration.
Ultimately, this interface for FAT32 file system can mount and identify file system, open and close files,
create and delete files, and R/W files.
The bottom sub-module has an identification ability to identify SDSC, SDHC / XC, and also MBR or DBR.

Through the practice of the subject, I have a deep knowledge and understanding of the FAT
file system and SD card. Meanwhile, the problem that  MCU shared information and data hard with PC by SD card
has been solved. It proves that MCU communicate with PC through formatted files is available.

\englishkeywords{SDC, MCU, FAT, file interface}
\end{englishabstract}

