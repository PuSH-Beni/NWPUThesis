%%==================================================
%% conclusion.tex for SJTUThesis
%% Encoding: UTF-8
%%==================================================

\chapter{全文总结}
\label{chap:Summary}

本文详细描述了基于微处理器的FAT32文件接口的设计与实现过程,
而该文件接口解决了单片机难以通过SD卡与计算机通信的问题,

通过对FAT文件系统和SD卡的SPI模式的学习和理解,阅读大量资料后,完成了文件接口的设计,
并最终在单片机上实现了该设计。
本文件接口在底层实现了单片机通过SPI总线协议与SD卡的数据传输,借助11脚SD卡槽,
能够对SD卡进行插拔检测,同时,能够识别并支持不同版本和容量的SD卡,无论是Version 2.0以后的卡还是Version 1.x的老式SD卡,无论是SDSC、SDHC还是SDXC均能正确识别并读写数据。
在SPI模式下,SPI总线时钟决定了数据的传输速率。本实验采用了C8051F020芯片,
其内置晶振最高$16Mhz$,而SPI的总线时钟最高为$8Mhz$,故在理想情况下\footnote{
    所有比特位都正确的传递,所有命令都能够无误收发},通过SPI协议与SD卡通信能够达到
的最大速率是$1MB/sec$\footnote{同前,此处的「MB」指1000 $\times$ 1000 Bytes}。

本文件接口最重要最核心的部分在于上层FAT模块的设计,该模块具有良好的可移植性,
通过隐藏不必要的程序接口实现了良好的稳定性。
该模块能够正确识别磁盘的「MBR」和「DBR」扇区,对于SD卡而言,通常只存在一个卷,且该卷
的起始扇区号被记录在整个磁盘的物理$0^{th}$扇区中,该FAT模块能够准确判断并迅速找到
SD卡的文件卷的真正起始扇区,不必一个扇区一个扇区地逐个判断,这加快了文件系统的挂载速度。
该模块实现了在根目录下对文件的创建、打开、写入、读取等一系列操作,同时支持「FSINFO」
数据结构,并且,通过「直写」(Write Through)的方式操作磁盘与内存的数据交换,
简化了结构的复杂度,实现了更好的兼容性,无需考虑同步问题。

当然,本工程仍然存在着待改进的地方,比如尚未实现对长文件名(LFN)的支持,
目前只能在根目录区操作等。

总之,本文件接口解决了单片机通过SD卡与计算机的数据共享问题,通过该接口可以
方便地向SD中存储数据或者读出数据,实现了兼容性,扩大了单片机的适用范围和用途。

