%%==================================================
%% chapter01.tex for NWPU Bachelor Thesis
%%==================================================
\chapter{绪论}
\label{chap:Introduction}
本课题是为解决单片机与个人计算机数据共享困难的问题而提出的。
本工程将为此设计并物理实现一个可行的、与FAT32文件系统相兼容的文件接口,它能够为单片机提供文件系统的部分功能,
实现对文件的管理和操作,对数据的组织和传输,同时,与计算机进行数据共享、文件互动也是其重要功能。
\section{背景}
\label{sec:Background}
绝大多数现在的单片机\supercite{mcu1,mcu2}(MCU)都是基于冯·诺伊曼结构的,这种结构清楚地定义了嵌入式系统所必需的四个基本部分:
一个中央处理器核心,程序存储器(只读存储器或者闪存)、数据存储器(随机存储器)、一个或者更多的定时/计数器,
还有用来与外围设备以及扩展资源进行通信的输入/输出端口——所有这些都被集成在单个集成电路芯片上。

单片机时钟频率通常较同时代的计算机芯片低,但它价格低廉,能够提供充足的程序存储器、丰富的片上接口。
某些架构的单片机生产厂商众多,例如8051系列\supercite{mcu3}、Z80系列。一些现代的微控制器支持一些内建的高级编程语言,
比如BASIC语言、C语言\supercite{mcu4}、C++等。

文件配置表(FAT),是一种由微软发明并拥有部分专利\footnote{专利申请在于文件系统中支持\textbf{长文件名\supercite{lfn1}}的技术,
而不是文件系统核心本身}的文件系统,供MS-DOS使用,也是所有非NT核心的微软窗口使用的文件系统。

相对于其他的文件系统,FAT文件系统\supercite{fatms}由于其使用简单的数据结构,使得文件操作耗时、磁盘空间利用率低等性能差的问题在存在许多小文件的情况下尤为突出\supercite{fat5}。
因为FAT文件系统考虑当时电脑性能有限,所以未被复杂化,这使它成为理想的软盘和存储卡文件系统,也适合用作不同操作系统中的数据交流。
传统的FAT32\supercite{fat4,fat8}文件系统能够兼容大容量存储器,所以仍然是便携式数字设备使用最广泛的文件系统\supercite{fatrv}。
\section{问题}
\label{sec:Issues}
如上所述,出于简化和成本的考虑,单片机往往没有配置SD卡\supercite{sdgp}接口,更不用说应的文件系统接口了,这导致单片机与PC计算机的数据共享变得困难。

一个没有文件系统的机器(单片机)和一个有文件系统的机器(计算机)是无法通过格式化的数据(Formatted Data)沟通的。
在没由相应的文件接口的情况下,单片机向SD卡写入的数据是无法被带有文件系统的计算机正确识别的。因为单片机不知道将数据写入到哪里,
也不知道如何组织数据,这不仅不能够正确地写入数据,还可能损坏原本的文件系统甚至损害磁盘媒介\footnote{比如将数据写入了磁盘的「MBR」扇区或者「DBR」扇区}。

反之,单片机从SD卡中读取数据时也无法正确识别按照特定的文件系统格式存放的数据,此时在单片机看来,SD卡中存储的数据是一些杂乱无章的字节。

\section{目标}
\label{sec:Goals}
通过设计并实现一个通用且兼容的文件系统接口(使用SD卡作为传输介质),将极大地方便单片机\supercite{fat6}与个人计算机的通信和数据交流。

此时本课题将设计一个文件系统\supercite{fatfs}接口,为那些自身不具备文件系统的机器(比如单片机),
将数据按照的特定的文件系统的格式组织管理好,同时提供文件系统的部分简化功能,做到与原生的文件系统完全兼容。

\section{方法}
\label{sec:Solution}
为了尽快完成对FAT32文件系统的设计实现和验证工作,本工程简化了不相关的因素(底层环境),故选择比较容易实现的SPI总线协议\supercite{spi1}作为单片机与SD通信\supercite{sdfat1,sdfat2}的方式,
即在SD卡的SPI模式下进行数据的物理传输。同时,微处理器选择了自带SPI硬件接口的C8051F020芯片,使之与一个11脚的SD卡槽相连接,实验平台就搭建完成了。

本文件接口实行模块化设计思路,将整个工程分为两层,各层相对独立,并且上层模块只能通过预留的API接口调用底层模块,底层模块不能主动与上层模块通信。

\section{意义}
\label{sec:Contributions}
FAT32文件系统接口的实现,使得单片机与个人计算机的数据共享变得可能,使得二者间的信息交流与通信方式更加多样化,同时,
通过使用便携式数字存储介质(比如SD卡、闪存盘、移动硬盘等),单片机可以向普通计算机一样,直观方便地组织和管理数据。

本文件接口能够实现的功能为:
\begin{itemize}[noitemsep,topsep=0pt,parsep=0pt,partopsep=0pt,leftmargin=5cm]
    \item SD卡的插入与拔出检测
    \item SDSC/SDHC/SDXC的识别
    \item 磁盘「MBR」扇区的识别
    \item 磁盘「DBR」扇区的定位
    \item FAT文件系统的识别与挂载
    \item 文件的创建与打开
    \item 文件的读出与写入
    \item 在文件的指定位置读写
    \item 文件的删除与关闭
\end{itemize}


\section{组织}
\label{sec:Orgnization}
本文由以下章节组成:
\begin{itemize}
    \item 第\ref{chap:Introduction}章,绪论,主要介绍了本课题的研究背景意义同时简单叙述了研究的整体思路与构想。
    \item 第\ref{chap:Design}章,设计,具体描述了FAT32文件系统接口的构造方式和设计思路。
    \item 第\ref{chap:Implementation}章,实现,具体描述了FAT32文件系统接口的物理实现过程,详细描述了具体函数的构造过程。
    \item 第\ref{chap:Results}章,实验与结果,简要描述了底层模块的实现与相应的实验验证环境,具体描述了本文件接口在具体实验中的验证和应用。
    \item 第\ref{chap:Summary}章,全文总结,对全文做以总结性的阐述。
\end{itemize}
